% resume.tex
% vim:set ft=tex spell:

\documentclass[11pt,a4paper]{article}
\usepackage[letterpaper,margin=0.5in]{geometry}
\usepackage{mdwlist}
\usepackage{fontspec}
\usepackage{textcomp}
\usepackage{tgpagella}
\usepackage{titlesec}
\pagestyle{empty}
\setlength{\tabcolsep}{0em}
\usepackage{graphicx}
\DeclareGraphicsExtensions{.pdf,.png,.jpg}

\setmainfont{Yantramanav}
\setsansfont{Apercu}
\titleformat*{\subsection}{\sffamily\Large}
% indentsection style, used for sections that aren't already in lists
% that need indentation to the level of all text in the document
\newenvironment{indentsection}[1]%
{\begin{list}{}%
	{\setlength{\leftmargin}{#1}}%
	\item[]%
}
{\end{list}}

% opposite of above; bump a section back toward the left margin
\newenvironment{unindentsection}[1]%
{\begin{list}{}%
	{\setlength{\leftmargin}{-0.5#1}}%
	\item[]%
}
{\end{list}}

% format two pieces of text, one left aligned and one right aligned
\newcommand{\headerrow}[2]
{\begin{tabular*}{\linewidth}{l@{\extracolsep{\fill}}r}
	#1 &
	#2 \\
\end{tabular*}}

% make "C++" look pretty when used in text by touching up the plus signs
\newcommand{\CPP}
{C\nolinebreak[4]\hspace{-.05em}\raisebox{.22ex}{\footnotesize\bf ++}}


% and the actual content starts here
\begin{document}

\begin{center}
% \includegraphics[scale=0.2]{./face2.jpg}\\
{\Huge \sffamily{Matt Bell}}\\


% \ \ Student at UCL  \textbullet
% \ \ Flat 6 Holmdale Mansions\ \ \textbullet
% \ \ London NW6 1BG
% \ \ \faicon{github} mbellgb\ \ \textbullet\
% \ \ \faicon{medium} @mbell\_gb\ \ \textbullet\
% \ \ \faicon{twitter} @mbell\_gb\ \ \textbullet\
% \\
% +44 7827 638207\ \ \textbullet\
\ \ matt@mbell.me\ \ \textbullet\
\ \ https://mbell.me/
\\
Twitter: @mbellgb
\end{center}


\hrule
\vspace{-1.2em}
\subsection*{Career}


 \begin{itemize}
 	\parskip=0.1em

 	\item
 	\headerrow%
 		{\textbf{State Street}}
		{\textbf{London, UK}}
 	\\
 	\headerrow%
 		{\emph{Microservices Framework Engineer}}
 		{\emph{Jul 2018 --- present}}
 	\begin{itemize*}
	    \item Part of core engineering work to provide Kubernetes-as-a-service, to eventually be rolled out across the business
	    \item Contributing to high level design of a highly-available, resilient platform as well as implementation details
	    \item Rolled out and maintained developer tools (Concourse CI and Artifactory) for internal users
	    \item Working amongst constraints of corporate and industry regulations to provide a secure, compliant framework for developers
 	\end{itemize*}

	\item
	\headerrow%
		{\textbf{Interrodata}}
		{\textbf{London, UK}}
	\\
	\headerrow%
		{\emph{Frontend Development (Contractor)}}
		{\emph{Aug 2017 --- Sep 2017}}
	\begin{itemize*}
	    \item Worked to integrate separate API backend (written in Python) with frontend app (written in React)
	    \item Set up automated build pipeline, including testing and code linting/formatting
	    \item Built new features for the frontend, including a sophisticated autocomplete system.
	\end{itemize*}

 	\item
 	\headerrow%
 		{\textbf{Pip App}}
		{\textbf{London, UK}}
 	\\
 	\headerrow%
 		{\emph{Backend Developer Intern (Contractor)}}
 		{\emph{Jun 2017 --- Jul 2017}}
 	\begin{itemize*}
	    \item Setup cloud data store (Firebase) and built admin interface/sync tool using Node.JS
	    \item Worked on integrating data into app, built on the React Native platform
 	\end{itemize*}

 	\item
 	\headerrow%
 		{\textbf{Netcraft}}
		{\textbf{Bath, UK}}
 	\\
 	\headerrow%
 		{\emph{Internet Services Developer (Internship)}}
 		{\emph{Jun 2016 --- Sep 2016}}
 	\begin{itemize*}
 		\item Maintained production servers and supported devops work, as well as working as the primary on-call systems admin at times.
		\item Develop new features for the company's systems including package version archiving and input validation for the fraud detection systems.
		\item Helped to set up clients on the internal systems and write config files for them.
 	\end{itemize*}

 \end{itemize}


 \vspace{-0.4em}
 \hrule
 \vspace{-1.2em}
 \subsection*{Education}

\begin{itemize}
	\parskip=0.1em

	\item
	\headerrow%
		{\textbf{University College London}}
		{\textbf{2:1 awarded}}\\
	\headerrow%
		{\emph{BSc Computer Science}}
		{\emph{Sep 2015 --- Jul 2018}}

\end{itemize}

\vspace{-0.4em}
\hrule
\vspace{-1.2em}

\subsection*{Technical Skills}

\begin{indentsection}{\parindent}
\hyphenpenalty=1000
\begin{description*}
	\item[Languages:]
	% JavaScript, C, Python, Perl, Rubya
    Expert: JavaScript, Ruby\\
    Working experience: Python, Go\\
    Intermediate: C\\
    Limited experience: Java
	\item[Web Technologies:]
	    Experience with popular web technologies such as Node.JS, Ruby On Rails, React and Django
	\item[Server-side engineering:]
	Experience using database systems (SQL and NoSQL), web server software (Nginx and Apache), UNIX shells and continuous integration systems (Concourse, Jenkins)
        \item[DevOps:] Experience building, running and maintaining Kubernetes services, Ansible, Terraform, AWS-based systems, Prometheus, limited ElasticSearch experience
	% I maintain my own virtual web server and have had experience setting up and running servers. I also have experience with web server software (Nginx \& Apache), database systems (PostgreSQL /MariaDB /MongoDB) and I am proficient with using UNIX shells.
\end{description*}
\end{indentsection}

% \vspace{-0.4em}
% \hrule
% \vspace{-1.2em}
% \subsection*{Achievements}
% \begin{indentsection}{\parindent}
% \hyphenpenalty=1000
% \begin{description*}
%     \item[Facebook Capture The Flag Competition (Nov 2015):] Worked on systems-level problems; my team took second place.
% 	\item[HackLondon II (Feb 2016):] Created a plugin for a text editor that allows gesture control, which won the GitHub developer prize.
% \end{description*}
% \end{indentsection}

\vspace{-0.4em}
\hrule
\vspace{-1.2em}
\subsection*{Volunteering and Hackathons}
\begin{indentsection}{\parindent}
\begin{description*}
	\item[UCLU Technology Society:] Was a committee member from March to October 2016, and a member of the organising team for the CS50 Hackathon.
	\item[Lauzhack 2016:] Travelled to Lausanne, Switzerland to volunteer/mentor at EPFL's hackathon. I helped with logistics and helped to fix attendees' projects.
	\item[Porticode 2016:] Mentored at UCL's beginner-focussed hackathon, helping specifically with Node.js and Git.
	\item[Hack The North 2017:] Travelled to Waterloo, Canada to hack with people from all over the world.
\end{description*}
\end{indentsection}
\vspace{-0.4em}
\hrule

\vspace{-1.2em}
\subsection*{Projects}
\begin{indentsection}{\parindent}
	\hyphenpenalty=1000
	\begin{description*}
		\item[Omnidash Hackathon Dashboard:]
		    A hackathon dashboard written in Ruby on Rails \& React. Allows attendees to view information about their event and find out about upcoming workshops, classes etc., whilst also allowing organisers to manage attendee invites and event information.
		\item[Final year project:]
		Worked on an IoT-based solution for remotely tracking and detecting wildlife species using a camera trap and deep learning.
		\item[UCL Assistant:] Worked on an open source, cross platform app (and API) for students at UCL to find study space and check their timetable. Uses React Native as well as plenty of interesting Node packages. Also uses a primitive form of caching using Redis.
	\end{description*}
	A selection of projects (plus source code) is available at https://mbell.me/projects.
\end{indentsection}

\end{document}
